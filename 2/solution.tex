\documentclass[11pt,a4paper]{scrartcl}
\usepackage[T1]{fontenc}
\usepackage[utf8]{inputenc}
\usepackage[ngerman]{babel}
\usepackage{microtype}
\usepackage{lmodern}
\usepackage{amsmath}
\usepackage{amsfonts}
\usepackage{amssymb}
\usepackage{enumerate}
\usepackage{graphicx}
\usepackage{listings}
\usepackage{color}
\usepackage{url}
\usepackage{multicol}
\usepackage{wrapfig}
\usepackage{amsmath}
\usepackage{tikz}
\usetikzlibrary{arrows}


%% http://stackoverflow.com/questions/741985/latex-source-code-listing-like-in-professional-books
\usepackage{courier}
\definecolor{light-gray}{gray}{0.95}
\lstset{
  % language=C,
  basicstyle=\small\sffamily,
%   basicstyle=\small\ttfamily,
  numbers=left,
  numberstyle=\tiny,
  frame=tb,
%  columns=fullflexible,
%  showstringspaces=false,
	backgroundcolor=\color{light-gray},
	linewidth=\linewidth,       % Zeilenbreite
	breaklines=true,             % Zeileumbruch
	breakatwhitespace=false, %Umbruch an Leerzeichen
  tabsize=2,
  extendedchars=true,
  xleftmargin=17pt,
  framexleftmargin=17pt,
  abovecaptionskip=7pt,
%   frameround=tttt,
}

\begin{document}

\author{Ralf Vogler}
\title{Program Optimization}
\subtitle{Exercise sheet 2}

\maketitle

\section*{Exercise 1: The language, CFGs, and semantics}
\subsection*{a) Draw the corresponding Control Flow Graph.}
\begin{tikzpicture}[->,>=stealth',shorten >=1pt,auto,node distance=1.5cm,
  thick,main node/.style={circle,fill=blue!20,draw,font=\sffamily\Large\bfseries}]

  \node[main node] (1) {0};
  \node[main node] [below of=1](2) {1};
  \node[main node] [below of=2](3) {2};
  \node[main node] [below of=3](4) {3};
  \node[main node] [below left of=4](5) {f};
  \node[main node] [below right of=4](6) {4};
  \node[main node] [below of=6](7) {5};
  \node[main node] [below of=7](8) {6};
  \node[main node] [below of=8](9) {7};

  \path[every node/.style={font=\sffamily\small}]
    (1) edge node [right] {c = 20;} (2)
    (2) edge node [right] {a = 2*c + 10;} (3)
    (3) edge node [right] {i = 0;} (4)
    (4) edge node [left] {Neg(i < n)} (5)
    (4) edge node [right] {Pos(i < n)} (6)
    (6) edge node [right] {b = 2*c + 10;} (7)
    (7) edge node [right] {a = a + b / c + 42;} (8)
    (8) edge node [right] {i = i+1;} (9)
    (9) edge [bend left] node {goto A;} (4);
\end{tikzpicture}

\subsection*{b) Write down the constraint system for calculating available expressions, give the set
of available expressions for each graph node.}

\begin{align*}
Constraints..C, Solution..S:\\
C
\left\{ \begin{array}{l}
\mathcal{A}[0] \subseteq \emptyset\\
\mathcal{A}[1] \subseteq (\mathcal{A}[0] \cup \{20\}) \backslash Expr_c\\
\mathcal{A}[2] \subseteq (\mathcal{A}[1] \cup \{2*c+10\}) \backslash Expr_a\\
\mathcal{A}[3] \subseteq (\mathcal{A}[2] \cup \{0\}) \backslash Expr_i\\
\mathcal{A}[3] \subseteq \mathcal{A}[7]\\
\mathcal{A}[4] \subseteq (\mathcal{A}[3] \cup \{i<n\})\\
\mathcal{A}[5] \subseteq (\mathcal{A}[4] \cup \{2*c+10\}) \backslash Expr_b\\
\mathcal{A}[6] \subseteq (\mathcal{A}[5] \cup \{a+b/c+42\}) \backslash Expr_a\\
\mathcal{A}[7] \subseteq (\mathcal{A}[6] \cup \{i+1\}) \backslash Expr_i\\
\mathcal{A}[f] \subseteq (\mathcal{A}[3] \cup \{i<n\})
\end{array}\right.
S
\left\{ \begin{array}{l}
\mathcal{A}[0] = \emptyset\\
\mathcal{A}[1] = \{20\}\\
\mathcal{A}[2] = \{20, 2*c+10\}\\
\mathcal{A}[3] = \{20, 2*c+10, 0\}\\
\mathcal{A}[4] = \{20, 2*c+10, 0, i<n\}\\
\mathcal{A}[5] = \{20, 2*c+10, 0, i<n\}\\
\mathcal{A}[6] = \{20, 2*c+10, 0, i<n\}\\
\mathcal{A}[7] = \{20, 2*c+10, 0\}\\
\mathcal{A}[f] = \{20, 2*c+10, 0, i<n\}
\end{array}\right.
%\left\{ \begin{array}{rcl} a \\ b \end{array}\right.
\end{align*}


\subsection*{c) Perform the transformation 1.1 and 1.2 to all non-trivial assignments in your graph.}
In the graph below only the important part has been changed (alternative to Tikz?).\\
Transformation 1.1 introduces storage registers $T_e$ for all non-trivial expressions $e$.\\
Transformation 1.2 replaces $e$ by $;$ (Nop) if $e$ is an available expression at node $u$.
If one looks at the set of available expressions for each node, this only affects the calculation of $2*c+10$ from node 4 to 5. So the edge $T_{2*c + 10} = 2*c + 10;$ can be replaced by $;$.
\begin{tikzpicture}[->,>=stealth',shorten >=1pt,auto,node distance=1.5cm,
  thick,main node/.style={circle,fill=blue!20,draw,font=\sffamily\Large\bfseries}]

  \node[main node] (1) {0};
  \node[main node] [below of=1](2) {1};
  \node[main node] [below of=2](3) {2};
  \node[main node] [below of=3](4) {3};
  \node[main node] [below left of=4](5) {f};
  \node[main node] [below right of=4](6) {4};
  \node[main node] [below of=6](7) {5};
  \node[main node] [below of=7](8) {6};
  \node[main node] [below of=8](9) {7};

  \path[every node/.style={font=\sffamily\small}]
    (1) edge node [right] {c = 20;} (2)
    (2) edge node [right] {a = 2*c + 10;} (3)
    (3) edge node [right] {i = 0;} (4)
    (4) edge node [left] {Neg(i < n)} (5)
    (4) edge node [right] {Pos(i < n)} (6)
    (6) edge node [right] {; b = $T_{2*c + 10}$;} (7)
    (7) edge node [right] {a = a + b / c + 42;} (8)
    (8) edge node [right] {i = i+1;} (9)
    (9) edge [bend left] node {goto A;} (4);
\end{tikzpicture}


\subsection*{d) Count the number of operations performed for n = 4, for the initial and the transformed case. One operation is a binary operation, an assignment to a variable, or an comparison. The statement x = y+1 counts as two operations.}
The transformed version doesn't have to reevaluate $T_{2*c + 10}$ and therefore saves $4*2$ operations.
If the unneccessary assignments to $b$ are also removed, it will save $1*2+3*3$ operations.
\begin{align*}
N_{initial} &= 5+4*(1+3+6)+1 = 46\\
N_{transformed} &= 5+4*(1+1+6)+1 = 38\\
N_{transformed_{improved}} &= 5+1+4*(1+0+6)+1 = 35
\end{align*}


\section*{Exercise 2: Constraint systems}
It is not a solution because it doesn't meet all the constraints:
\begin{align*}
&\mathcal{A}[0] = \emptyset \subseteq \emptyset \rightarrow ok\\
&\mathcal{A}[1] = \{1,2,3\} \subseteq (\emptyset \cup \{1,2\}) \backslash \{4,5\} &\rightarrow \mbox{fail because } \{1,2,3\} \nsubseteq \{1,2\}\\
\end{align*}
$\mathcal{A}[1]$ should be $\{1\}$ because $\mathcal{A}[1] \subseteq \{1,2\} \cap \{1,3\}$.

\section*{Exercise 3: Least fixpoint}
The least fixpoint for $f$ is 100 because $f(100) = min(101,100) = 100$. It must be the least fixpoint because it is the only fixpoint.
\begin{align*}
f(x) = \left\{ \begin{array}{lcl}
x+1, &\mbox{if } x<100 &\rightarrow f(x) \neq x\\
100, &\mbox{otherwise} &\rightarrow f(x) = x \mbox{ iff } x = 100
\end{array}\right.
\end{align*}


\section*{Exercise 4: Monotonicity}
\begin{align*}
&\textbf{dec }a \leq \textbf{dec }b\\
\Leftarrow &\text{ \{apply the definition of dec\}}\\
&a-1 \leq b-1\\
\Leftarrow &\text{ \{move the ''-1'' to the right\}}\\
&a \leq b-1+1\\
\Leftarrow &\text{ \{simplify\}}\\
&a \leq b
\end{align*}

\begin{align*}
&\textbf{three }a \leq \textbf{three }b\\
\Leftarrow &\text{ \{apply the definition of three\}}\\
&3 \leq 3\\
\Leftarrow &\text{ \{simplify\}}\\
&true\\
\Leftarrow &a \leq b \Rightarrow true \Leftrightarrow true\\
&a \leq b
\end{align*}

\begin{align*}
&\textbf{f }a \leq \textbf{f }b\\
\Leftarrow &\text{ Part 1: \{f(x) = 0 is monotonic for x < 4\}}\\
&0 \leq 0\\
\Leftarrow &\text{ Part 2: \{show that $x^2$ is monotonic for x $\geq$ 4\}}\\
&a^2 \leq b^2\\
\Leftarrow &\text{ \{$b = a+x$ with $x \geq 0$\}}\\
&a^2 \leq (a+x)^2\\
\Leftarrow &\text{ \{simplify\}}\\
&a^2 \leq a^2+2*a*x+x^2\\
\Leftarrow &\text{ \{simplify\}}\\
&0 \leq 2*a*x+x^2\\
\Leftarrow &\text{ \{simplify\}}\\
&a \leq b
\end{align*}


\begin{align*}
&\textbf{inv }a \leq \textbf{inv }b\\
\Leftarrow &\text{ \{apply the definition of inv\}}\\
&-a \leq -b\\
\Leftarrow &\text{ \{multiply with ''-1''\}}\\
&a > b
\end{align*}

\begin{align*}
&a \leq b\\
\Rightarrow &\text{ \{let $a=3$ and $b=4$\}}\\
&\textbf{g }3 \leq \textbf{g }4\\
\Rightarrow &\text{ \{apply the definition of g\}}\\
&40 \nleq 4^2\\
\Rightarrow &\text{ \{fails\}}\\
&\textbf{g }a \nleq \textbf{g }b
\end{align*}

\end{document}

















