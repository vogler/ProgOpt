\documentclass[11pt,a4paper]{scrartcl}
\usepackage[T1]{fontenc}
\usepackage[utf8]{inputenc}
%\usepackage[ngerman]{babel}
\usepackage[ngerman,english]{babel}
\selectlanguage{english}
\usepackage{microtype}
\usepackage{lmodern}
\usepackage{amsmath}
\usepackage{amsfonts}
\usepackage{amssymb}
\usepackage{enumerate}
\usepackage{graphicx}
\usepackage{listings}
\usepackage{color}
\usepackage{url}
\usepackage{multicol}
\usepackage{wrapfig}
\usepackage{amsmath}
\usepackage{tikz}
\usetikzlibrary{arrows}


\begin{document}

\author{Ralf Vogler}
\title{Program Optimization}
\subtitle{Exercise sheet 3}

\maketitle

\section*{Exercise 1: Solving}
\begin{minipage}{0.4\textwidth}
\begin{align*}
&\mathcal{A}[0] \subseteq \emptyset\\
&\mathcal{A}[1] \subseteq (\mathcal{A}[0] \cup \{1,2\}) \backslash \{4,5\}\\
&\mathcal{A}[1] \subseteq (\mathcal{A}[4] \cup \{1,3\}) \backslash \{5\}\\
&\mathcal{A}[2] \subseteq \mathcal{A}[1] \backslash \{1,3\}\\
&\mathcal{A}[3] \subseteq (\mathcal{A}[2] \cup \{4,5\}) \backslash \{2,5\}\\
&\mathcal{A}[4] \subseteq (\mathcal{A}[3] \cup \{1\}) \backslash \{2,3\}\\
\end{align*}
\end{minipage}
\begin{table}[h]
\begin{minipage}{0.4\textwidth}
\begin{tabular}{c|c|c|c|c}
$\mathcal{A}[i]$ & 0 & 1 & 2 & 3 \\ \hline
$\mathcal{A}[0]$ & $\emptyset$ & $\emptyset$ & $\emptyset$ & dito \\
$\mathcal{A}[1]$ & $\emptyset$ & $\{1\}$ & $\{1\}$ \\
$\mathcal{A}[2]$ & $\emptyset$ & $\emptyset$ & $\emptyset$ \\
$\mathcal{A}[3]$ & $\emptyset$ & $\{4\}$ & $\{4\}$ \\
$\mathcal{A}[4]$ & $\emptyset$ & $\{1\}$ & $\{1,4\}$ \\
\end{tabular}
\caption{Naive iteration}
\end{minipage}
\begin{minipage}{0.3\textwidth}
\begin{tabular}{c|c|c|c}
$\mathcal{A}[i]$ & 0 & 1 & 2 \\ \hline
$\mathcal{A}[0]$ & $\emptyset$ & $\emptyset$ & dito \\
$\mathcal{A}[1]$ & $\emptyset$ & $\{1\}$ \\
$\mathcal{A}[2]$ & $\emptyset$ & $\emptyset$ \\
$\mathcal{A}[3]$ & $\emptyset$ & $\{4\}$ \\
$\mathcal{A}[4]$ & $\emptyset$ & $\{1,4\}$ \\
\end{tabular}
\caption{RR iteration}
\end{minipage}
\end{table}

\section*{Exercise 2: RR is Correct!}
\begin{enumerate}
\item \begin{align}
&y_i^{(d)} \sqsubseteq x_i^{(d)}\\
\rightarrow &F^d \perp \sqsubseteq RR^d \perp\\
\rightarrow \text{induction on $d$:}\\
\text{Foundation $(d = 0)$}&: \hspace{1cm} y_i^{(0)} = F^0 \perp = \perp \sqsubseteq  RR^0 \perp = x_i^{(0)}\\
\text{Step $(d+1)$ using (2)}&: \hspace{1cm} F^{d+1} \perp = F(F^d \perp) \sqsubseteq F(RR^d \perp) \sqsubseteq  RR(RR^d \perp) = RR^{d+1} \perp
\end{align}
\item \begin{align*}
x_i^{(d)} \sqsubseteq z_i^{(d)} \text{ for every solution } (z_1,...,z_n)
\end{align*}
holds because $x_i^{(d)}$ will be the least solution among all solutions $z$ that satisfy $z = F(z)$ (slides 107ff). This is the case because $y_i^{(d)} \sqsubseteq z_i^{(d)}$ and $y_i^{(d)} \sqsubseteq x_i^{(d)}$, which means that RR iteration will reach the same (least) solution as naive iteration, only faster.
\end{enumerate}


\section*{Exercise 3: Merge/Meet Over all Paths}
\begin{minipage}{0.35\textwidth}
\begin{tikzpicture}[->,>=stealth',shorten >=1pt,auto,node distance=1.8cm,
  thick,main node/.style={circle,fill=blue!20,draw,font=\sffamily\Large\bfseries}]

  \node[main node] (1) {0};
  \node[main node] [below right of=1](2) {1};
  \node[main node] [below of=2](3) {2};
  \node[main node] [below of=3](4) {3};
  \node[main node] [below left of=1](5) {4};
  \node[main node] [below of=5](6) {5};
  \node[main node] [below of=6](7) {6};
  \node[main node] [below right of=7](8) {7};
  \node[main node] [below of=8](9) {f};

  \path[every node/.style={font=\sffamily\small}]
    (1) edge node [right] {Pos(x)} (2)
    (2) edge node [right] {x = 1;} (3)
    (3) edge node [right] {y = 2;} (4)
    (1) edge node [left] {Neg(x)} (5)
    (5) edge node [right] {y = 1;} (6)
    (6) edge node [right] {x = 2;} (7)
    (7) edge node [left] {;} (8)
    (4) edge node [right] {;} (8)
    (8) edge node [right] {x = x * y} (9);
\end{tikzpicture}
\end{minipage}
\begin{minipage}{0.65\textwidth}
\begin{align*}
\mathcal{I}^*[0] &= d_0 = (\mathbb{N}, \mathbb{N})\\
\mathcal{I}^*[1] &= [\![Pos(x)]\!]^\# d_0 = (\mathbb{N}, \mathbb{N})\\
\mathcal{I}^*[2] &= (\{1\}, \mathbb{N})\\
\mathcal{I}^*[3] &= (\{1\}, \{2\})\\
\mathcal{I}^*[4] &= [\![ Neg(x)]\!]^\# d_0 = (\mathbb{N}, \mathbb{N})\\
\mathcal{I}^*[5] &= (\mathbb{N}, \{1\})\\
\mathcal{I}^*[6] &= (\{2\}, \{1\})\\
\mathcal{I}^*[7] &= \bigsqcup \{(\{1\}, \{2\}), (\{2\}, \{1\})\} = (\{1,2\}, \{1,2\})\\
\mathcal{I}^*[f] &= (\{1*1, 1*2, 2*1, 2*2\}, \{1,2\}) = (\{1, 2, 4\}, \{1,2\})\\
\end{align*}
\end{minipage}\\ \\


\section*{Exercise 4: Live variables}
\begin{minipage}{0.3\textwidth}
\begin{tikzpicture}[->,>=stealth',shorten >=1pt,auto,node distance=1.8cm,
  thick,main node/.style={circle,fill=blue!20,draw,font=\sffamily\Large\bfseries}]

  \node[main node] (1) {0};
  \node[main node] [below of=1](2) {1};
  \node[main node] [below of=2](3) {2};
  \node[main node] [below of=3](4) {3};
  \node[main node] [below of=4](5) {4};

  \path[every node/.style={font=\sffamily\small}]
    (1) edge node [right] {x = y+4;} (2)
    (2) edge node [right] {y = 2;} (3)
    (3) edge node [right] {x = y+7;} (4)
    (4) edge node [right] {M[y] = x;} (5);
\end{tikzpicture}
\end{minipage}
\begin{minipage}{0.4\textwidth}
Constraints:
\begin{align*}
\mathcal{L}[0] &\supseteq (\mathcal{L}[1] \backslash \{x\}) \cup \{y\}\\
\mathcal{L}[1] &\supseteq \mathcal{L}[2] \backslash \{y\}\\
\mathcal{L}[2] &\supseteq (\mathcal{L}[3] \backslash \{x\}) \cup \{y\}\\
\mathcal{L}[3] &\supseteq \mathcal{L}[4] \cup \{y, x\}\\
\mathcal{L}[4] &\supseteq \emptyset \\
\end{align*}
\end{minipage}
\begin{minipage}{0.3\textwidth}
Solution:
\begin{align*}
\mathcal{L}[0] &= \{y\} \\
\mathcal{L}[1] &= \emptyset \\
\mathcal{L}[2] &= \{y\} \\
\mathcal{L}[3] &= \{x, y\} \\
\mathcal{L}[4] &= \emptyset \\
\end{align*}
\end{minipage}

\end{document}

















