\documentclass[11pt,a4paper]{scrartcl}
\usepackage[T1]{fontenc}
\usepackage[utf8]{inputenc}
%\usepackage[ngerman]{babel}
\usepackage[ngerman,english]{babel}
\selectlanguage{english}
\usepackage{microtype}
\usepackage{lmodern}
\usepackage{amsmath}
\usepackage{amsfonts}
\usepackage{amssymb}
\usepackage{enumerate}
\usepackage{graphicx}
\usepackage{listings}
\usepackage{color}
\usepackage{url}
\usepackage{multicol}
\usepackage{wrapfig}
\usepackage{amsmath}
\usepackage{tikz}
\usetikzlibrary{arrows}


\begin{document}

\author{Ralf Vogler}
\title{Program Optimization}
\subtitle{Exercise sheet 5}

\maketitle

\section*{Exercise 1: More constant propagation}
\begin{minipage}[t]{0.5\textwidth}
\vspace{0pt}
\subsection*{1. Draw CFG}
\begin{tikzpicture}[->,>=stealth',shorten >=1pt,auto,node distance=1.8cm,
  thick,main node/.style={circle,fill=blue!20,draw,font=\sffamily\Large\bfseries}]

  \node[main node] (1) {0};
  \node[main node] [below of=1] (2) {1};
  \node[main node] [below of=2] (3) {2};
  \node[main node] [below of=3] (4) {3};
  \node[main node] [below left of=4](5) {4};
  \node[main node] [below right of=4](6) {5};
  \node[main node] [below of=6] (7) {6};
  \node[main node] [below of=7] (8) {7};
  \node[main node] [below of=5] (9) {f};

  \path[every node/.style={font=\sffamily\small}]
    (1) edge node [right] {x = 0;} (2)
    (2) edge node [right] {y = 1;} (3)
    (3) edge node [right] {z = 1;} (4)
    (4) edge node [left] {Neg(z <= y)} (5)
    (4) edge node [right] {Pos(z <= y)} (6)
    (6) edge node [right] {y = x*z + y;} (7)
    (7) edge node [right] {x = y - 1;} (8)
    (8) edge [bend left] node {z = z*2;} (4)
    (5) edge node [left] {r = y;} (9);
\end{tikzpicture}
\end{minipage}
\begin{minipage}[t]{0.5\textwidth}
\vspace{0pt}
\subsection*{2. Contraints}
\begin{align*}
&\mathcal{D}[0] \sqsubseteq \{x \mapsto \top, y \mapsto \top, z \mapsto \top, r \mapsto \top\} = \mathcal{D}_{\top}\\
&\mathcal{D}[1] \sqsubseteq \mathcal{D}[0] \oplus \{x \mapsto 0\}\\
&\mathcal{D}[2] \sqsubseteq \mathcal{D}[1] \oplus \{y \mapsto 1\}\\
&\mathcal{D}[3] \sqsubseteq \mathcal{D}[2] \oplus \{z \mapsto 1\}\\
&\mathcal{D}[3] \sqsubseteq \mathcal{D}[7] \oplus \{z \mapsto \mathcal{D}[7](z)*2\}\\
&\mathcal{D}[4] \sqsubseteq \mathcal{D}[3]\\
&\mathcal{D}[f] \sqsubseteq \mathcal{D}[4] \oplus \{r \mapsto \mathcal{D}[4](y)\}\\
&\mathcal{D}[5] \sqsubseteq \mathcal{D}[3]\\
&\mathcal{D}[6] \sqsubseteq \mathcal{D}[5] \oplus \{y \mapsto \mathcal{D}[5](x)*\mathcal{D}[5](z)+\mathcal{D}[5](y)\}\\
&\mathcal{D}[7] \sqsubseteq \mathcal{D}[6] \oplus \{x \mapsto \mathcal{D}[6](y)-1\}
\end{align*}
\end{minipage}

\begin{minipage}[t]{0.6\textwidth}
\vspace{0pt}
\subsection*{3a. Solution}
Iteration 1:
\begin{align*}
\mathcal{D}[0] &= \{x \mapsto \top, y \mapsto \top, z \mapsto \top, r \mapsto \top\}\\
\mathcal{D}[1] &= \{x \mapsto 0, y \mapsto \top, z \mapsto \top, r \mapsto \top\}\\
\mathcal{D}[2] &= \{x \mapsto 0, y \mapsto 1, z \mapsto \top, r \mapsto \top\}\\
\mathcal{D}[3] &= \{x \mapsto 0, y \mapsto 1, z \mapsto 1, r \mapsto \top\}\\
\mathcal{D}[5] &= \mathcal{D}[3]\\
\mathcal{D}[6] &= \{x \mapsto 0, y \mapsto 1, z \mapsto 1, r \mapsto \top\}\\
\mathcal{D}[7] &= \{x \mapsto 0, y \mapsto 1, z \mapsto 1, r \mapsto \top\}\\
\mathcal{D}[4] &= \top\\
\mathcal{D}[f] &= \top
\end{align*}
Iteration 2:
\begin{align*}
\mathcal{D}[0] &= "\\
\mathcal{D}[1] &= "\\
\mathcal{D}[2] &= "\\
\mathcal{D}[3] &= \mathcal{D}[3] \sqcup \{x \mapsto 0, y \mapsto 1, z \mapsto 2, r \mapsto \top\}\\
&= \{x \mapsto 0 \sqcup 0, y \mapsto 1 \sqcup 1, z \mapsto 1 \sqcup 2, r \mapsto \top \sqcup \top\}\\
&= \{x \mapsto 0, y \mapsto 1, z \mapsto \top, r \mapsto \top\}\\
\mathcal{D}[5] &= \mathcal{D}[3]\\
\mathcal{D}[6] &= \{x \mapsto 0, y \mapsto 1, z \mapsto \top, r \mapsto \top\}\\
\mathcal{D}[7] &= \{x \mapsto 0, y \mapsto 1, z \mapsto \top, r \mapsto \top\}\\
\mathcal{D}[4] &= \mathcal{D}[3]\\
\mathcal{D}[f] &= \{x \mapsto 0, y \mapsto 1, z \mapsto \top, r \mapsto 1\}
\end{align*}
\end{minipage}
\begin{minipage}[t]{0.4\textwidth}
\subsection*{3b. Simplified solution}
\begin{tikzpicture}[->,>=stealth',shorten >=1pt,auto,node distance=1.8cm,
  thick,main node/.style={circle,fill=blue!20,draw,font=\sffamily\Large\bfseries}]

  \node[main node] (1) {0};
  \node[main node] [below of=1] (2) {1};
  \node[main node] [below of=2] (3) {2};
  \node[main node] [below of=3] (4) {3};
  \node[main node] [below left of=4](5) {4};
  \node[main node] [below right of=4](6) {5};
  \node[main node] [below of=6] (7) {6};
  \node[main node] [below of=7] (8) {7};
  \node[main node] [below of=5] (9) {f};

  \path[every node/.style={font=\sffamily\small}]
    (1) edge node [right] {;} (2)
    (2) edge node [right] {;} (3)
    (3) edge node [right] {;} (4)
    (4) edge node [left] {;} (5)
    (4) edge [dashed] node [right] {;} (6)
    (6) edge [dashed] node [right] {;} (7)
    (7) edge [dashed] node [right] {;} (8)
    (8) edge [dashed, bend left] node {;} (4)
    (5) edge node [left] {r = 1;} (9);
\end{tikzpicture}
\end{minipage}


\section*{Exercise 2: Description relation}
\begin{itemize}
\item Abstract (ignore $\mu$):
\begin{align*}
(\{x \mapsto 2, y \mapsto 4\})\\
(\{x \mapsto 2, y \mapsto \top\})\\
(\{x \mapsto 2, \top \mapsto 4\})\\
(\{x \mapsto \top, y \mapsto \top\})
\end{align*}
\item Concrete:
\begin{align*}
(\{x \mapsto 2, y \mapsto 4\})\\
(\{x \mapsto 2, y \mapsto 5\})\\
(\{x \mapsto 3, y \mapsto 4\})\\
(\{x \mapsto 3, y \mapsto 5\})\\
\end{align*}
\end{itemize}



\section*{Exercise 3: Intervals}
\subsection*{a)}
$[0,0] *^\sharp ([2,3] <^\sharp [1,5]) = [0,0] *^\sharp [0,1] = [0,0]$

\subsection*{b)}
$(-[2,3]) *^\sharp (-[2,3]) = [-3,-2] *^\sharp [-3,-2] = [4 \sqcap 6 \sqcap 9, 4 \sqcup 6 \sqcup 9] = [4,9]$

\subsection*{c)}
\begin{align*}
&[1,5] ==^\sharp ([1,5] +^\sharp [1,1] -^\sharp [1,1]) = \\
&[1,5] ==^\sharp ([2,6] +^\sharp (-^\sharp [1,1])) = \\
&[1,5] ==^\sharp ([2,6] +^\sharp [-1,-1]) = \\
&[1,5] ==^\sharp [1,5] = [0,1]
\end{align*}

\subsection*{d)}
$([1,5] +^\sharp [2,3]) /^\sharp [2,2] = [3,8] /^\sharp [2,2] = [\dfrac{3}{2} \sqcap \dfrac{8}{2}, \dfrac{3}{2} \sqcup \dfrac{8}{2}] = [\dfrac{3}{2}, 4]$


\section*{Exercise 4: Optimizing abstract operations}
\subsection*{a) More precise $*^\sharp$}
Old definition:
\begin{align*}
a *^\sharp b := \left\{ \begin{array}{ll}
\top &\mbox{if } a=\top \mbox{ or } b=\top \\
a*b  &\mbox{otherwise}
\end{array}\right.
\end{align*}
Optimized:
\begin{align*}
a *^\sharp b := \left\{ \begin{array}{ll}
0 &\mbox{if } a=0 \mbox{ or } b=0 \\
a*b  &\mbox{if } a \in \mathbb{Z} \backslash \{0\} \\
\top &\mbox{otherwise}
\end{array}\right.
\end{align*}

\subsection*{b) Imprecise abstract semantics}
\begin{enumerate}
\item \begin{align*}
&e := x==x; \hspace*{1em} d := \{x \mapsto \top\} \\
&\gamma([\![e]\!]^\sharp d) = \gamma(\top) = \mathbb{Z} \\
\neq &\{[\![e]\!] s \; | \; s \in \gamma(d)\} = \{1\}
\end{align*}

\item \begin{align*}
&e := x-x; \hspace*{1em} d := \{x \mapsto \top\} \\
&\gamma([\![e]\!]^\sharp d) = \gamma(\top) = \mathbb{Z} \\
\neq &\{[\![e]\!] s \; | \; s \in \gamma(d)\} = \{0\}
\end{align*}
\end{enumerate}

\end{document}

















