\documentclass[11pt,a4paper]{scrartcl}
\usepackage[T1]{fontenc}
\usepackage[utf8]{inputenc}
\usepackage[ngerman]{babel}
\usepackage{microtype}
\usepackage{lmodern}
\usepackage{amsmath}
\usepackage{amsfonts}
\usepackage{amssymb}
\usepackage{enumerate}
\usepackage{graphicx}
\usepackage{listings}
\usepackage{color}
\usepackage{url}
\usepackage{multicol}
\usepackage{wrapfig}
\usepackage{amsmath}
\usepackage{tikz}
\usetikzlibrary{arrows}


%% http://stackoverflow.com/questions/741985/latex-source-code-listing-like-in-professional-books
\usepackage{courier}
\definecolor{light-gray}{gray}{0.95}
\lstset{
  % language=C,
  basicstyle=\small\sffamily,
%   basicstyle=\small\ttfamily,
  numbers=left,
  numberstyle=\tiny,
  frame=tb,
%  columns=fullflexible,
%  showstringspaces=false,
	backgroundcolor=\color{light-gray},
	linewidth=\linewidth,       % Zeilenbreite
	breaklines=true,             % Zeileumbruch
	breakatwhitespace=false, %Umbruch an Leerzeichen
  tabsize=2,
  extendedchars=true,
  xleftmargin=17pt,
  framexleftmargin=17pt,
  abovecaptionskip=7pt,
%   frameround=tttt,
}

\begin{document}

\author{Ralf Vogler}
\title{Program Optimization}
\subtitle{Exercise sheet 1}

\maketitle

\section*{Exercise 1: The language, CFGs, and transformations}
\subsection*{a) Translate the program to our simplified imperative language.}
\begin{lstlisting}
r = 0;							// int r = 0;
i = 0;							// int i = 0;
A: if(i < n){				// for-loop condition
	A_1 = A_0 + 1*i;	// A_0 == &a and assuming element size 1
	r = r + M[A_1];		// r += a[i];
	i = i + 1;				// i++
	goto A;						// continue
} else { }
R_0 = r;						// return r;
	
\end{lstlisting}


\subsection*{b) Draw the corresponding Control Flow Graph.}
\begin{tikzpicture}[->,>=stealth',shorten >=1pt,auto,node distance=1.5cm,
  thick,main node/.style={circle,fill=blue!20,draw,font=\sffamily\Large\bfseries}]

  \node[main node] (1) {0};
  \node[main node] [below of=1](2) {1};
  \node[main node] [below of=2](3) {2};
  \node[main node] [below left of=3](4) {f};
  \node[main node] [below right of=3](5) {3};
  \node[main node] [below of=5](6) {4};
  \node[main node] [below of=6](7) {5};
  \node[main node] [below of=7](8) {6};

  \path[every node/.style={font=\sffamily\small}]
    (1) edge node [right] {r = 0;} (2)
    (2) edge node [right] {i = 0;} (3)
    (3) edge node [left] {Neg(i < n)} (4)
    (5) edge node [right] {$A_1$ = $A_0$ + 1*i;} (6)
    (6) edge node [right] {r = r + M[$A_1$];} (7)
    (7) edge node [right] {i = i + 1;} (8)
    (8) edge [bend left] node {goto A;} (3)
    (3) edge node [right] {Pos(i < n)} (5);
\end{tikzpicture}


\subsection*{c) Write down the path ($\pi = ...$) that represents the whole function call with n = 1.}
$\pi = r = 0;
\\i = 0;
\\Pos(i < n);
\\A_1 = A_0 + 1*i;
\\r = r + M[A_1];
\\i = i+1;
\\goto\;A;
\\Neg(i < n);$

\subsection*{d) Perform the transformation 1.1 to all non-trivial assignments in your graph.
All assignments of just one constant or one variable are, in this exercise, considered
trivial.}
\begin{tikzpicture}[->,>=stealth',shorten >=1pt,auto,node distance=1.5cm,
  thick,main node/.style={circle,fill=blue!20,draw,font=\sffamily\Large\bfseries}]

  \node[main node] (1) {0};
  \node[main node] [below of=1](2) {1};
  \node[main node] [below of=2](9) {2a};
  \node[main node] [below of=9](3) {2b};
  \node[main node] [below left of=3](4) {f};
  \node[main node] [below right of=3](5) {3a};
  \node[main node] [below of=5](10) {3b};
  \node[main node] [below of=10](6) {4};
  \node[main node] [below of=6](7) {5};
  \node[main node] [below of=7](8) {6};

  \path[every node/.style={font=\sffamily\small}]
    (1) edge node [right] {r = 0;} (2)
    (2) edge node [right] {i = 0;} (9)
    (9) edge node [right] {$T_{i < n}$ = i < n;} (3)
    (3) edge node [left] {Neg($T_{i < n}$)} (4)
    (5) edge node [right] {$T_{A_0 + i;}$ = $A_0$ + i;} (10)
    (10) edge node [right] {$A_1$ = $T_{A_0 + i;}$} (6)
    (6) edge node [right] {r = r + M[$A_1$];} (7)
    (7) edge node [right] {i = i + 1;} (8)
    (8) edge [bend left] node {goto A;} (9)
    (3) edge node [right] {Pos($T_{i < n}$)} (5);
\end{tikzpicture}\\
The transformation isn't done for $i = i + 1$ and $r = r + M[A_1];$ because $x \in Vars(e)$ (cf. slide 56). Here the transformation doesn't result in any Nop-steps because the expressions in the loop depend on $i$ which changes and therefore removes them from the set of available expressions.

\section*{Exercise 2: Transformations}
$\rho = \{x \mapsto \mu(2), y \mapsto 0, z \mapsto 0\}$\\
$\mu = \{2 \mapsto 42\}$

\end{document}